\documentclass{article} % For LaTeX2e
\usepackage{nips15,times}

\usepackage[colorlinks=true,
linkcolor=blue,
urlcolor=blue]{hyperref}

\usepackage{graphicx}
%\documentstyle[nips14submit_09,times,art10]{article} % For LaTeX 2.09
\title{Research Statement -- Shafiq Joty}
\newcommand{\fix}{\marginpar{FIX}}
\newcommand{\new}{\marginpar{NEW}}
% DEFINITIONS of the paper (use \newcommand instead of \def)

%%% inline lists
\newcommand{\Ni}{({\em i})~}
\newcommand{\Nii}{({\em ii})~}
\newcommand{\Niii}{({\em iii})~}
\newcommand{\Niv}{({\em iv})~}
\newcommand{\Nv}{({\em v})~}
\newcommand{\Nvi}{({\em vi})~}


\newcommand{\Na}{({\em a})~}
\newcommand{\Nb}{({\em b})~}
\newcommand{\Nc}{({\em c})~}


\newcommand{\srccon}{({\em $S_i$})~}

%%% MT metrics and variants
\newcommand{\qcri}{\nobreak{DR}}
\newcommand{\qcril}{\nobreak{DR-{\sc lex}}}
\newcommand{\drLEXe}{\nobreak{DR-{\sc lex}$_e$}}
\newcommand{\sqcri}{\nobreak{\sc dr}}
\newcommand{\sqcril}{\nobreak{\sc dr-{\scriptsize lex}}}
\newcommand{\full}{\nobreak{\emph{full}}}
\newcommand{\nor}{\nobreak{\emph{no\_rel}}}
\newcommand{\non}{\nobreak{\emph{no\_nuc}}}
\newcommand{\nonr}{\nobreak{\emph{no\_nuc \& no\_rel}}}
\newcommand{\nostr}{\nobreak{\emph{no\_discourse}}}

\newcommand{\fo}{\nobreak{F$_1$}}
\newcommand{\rmse}{\nobreak{RMSE}}

\newcommand{\dummy}{\nobreak{\emph{dummy}\hspace{3pt}}}


% DR-based individual metrics
\newcommand{\drLEX}{\nobreak{DR-{\sc lex}$_{1}$}}
\newcommand{\dr}{\nobreak{DR-{\sc nolex}}}
\newcommand{\drLEXto}{\nobreak{DR-{\sc lex}$_{2}$}}
\newcommand{\drLEXoo}{\nobreak{DR-{\sc lex}$_{1.1}$}}
\newcommand{\drLEXot}{\nobreak{DR-{\sc lex}$_{1.2}$}}
%\newcommand{\drLEXtt}{\nobreak{DR-{\sc lex}$_{2.1}$}}

% DiscoTK metrics
\newcommand{\disco}{\nobreak{DiscoTK}}
\newcommand{\discolight}{\nobreak{{\sc DiscoTK}$_{light}$}}
\newcommand{\discoparty}{\nobreak{{\sc DiscoTK}$_{party}$}}

\nipsfinalcopy % Uncomment for camera-ready version

\begin{document}

\maketitle

The Internet is a great source of human knowledge, but most of the information is in the form of unstructured text. In Natural Language Processing (NLP), we focus on adding structure to this text to uncover relevant information, and to use it in developing end-user application programs. To this end, my primary research goal is twofold: \Ni  developing NLP tools to automatically understand language phenomena that go beyond the individual clauses or sentences of a text, i.e., the discourse structure of the text; and \Nii exploiting these discourse analysis tools effectively in downstream NLP applications including machine translation, summarization, question answering, and sentiment analysis. One methodology emphasized throughout my research is to first identify the inherent semantic structures in a given problem, and then to develop structured machine learning models to exploit such structures effectively. My work has relied on deep learning for better representation of the input text and on probabilistic graphical models for capturing dependencies in the output.     
 

A significant part of my discourse research focuses on a special form of discourse called \emph{asynchronous conversations}, i.e., conversations where participants communicate with each other at different times (e.g., forums, emails, twitter). While the number of applications targeting these conversations is growing, sophisticated NLP tools to analyze these conversations are still not sufficiently accurate to support those applications. Also, tools developed for analyzing monologues (e.g., news articles) are not as effective when applied directly to these conversations because the two forms of discourse are different in many aspects. My PhD thesis focused on building novel computational models for different discourse analysis tasks in monologues and in asynchronous conversations, which I describe in Section \ref{discourse}. After PhD, I started working on the applications of discourse analysis, which are described in Section \ref{discourse_app}. Apart from discourse and its applications, I have also developed novel machine learning models for various NLP applications, which are highlighted in Section \ref{application_ML}. 

I am also interested in multidisciplinary research that goes beyond NLP, and have developed predictive models for a number of data mining tasks in social networks and in health science. I have recently embarked on joint research projects involving multiple research groups, where my collaborators and I are investigating multilingual (e.g., Mandarine, Tamil) and multimodal (e.g., image, video) language processing problems. Some of these efforts are described in Section \ref{application_ML}. Plans for future research that are directly related to my previous work are noted in the respective sections. I have a number of other ideas for future research directions, which I discuss in Section \ref{future}.       

% among three research groups at QCRI --  language technologies, social computing and data analytics -- where my collaborators and I are investigating machine learning models for group mining tasks in social networks and for prediction tasks in crisis computing.  



  

%Aside from discourse analysis and its applications, I am also  interested in developing effective machine learning  models for end-user applications. My interests lie in two important sub-fields of machine learning, deep learning and probabilistic graphical models, and their combination.  
 



%in §1 and §2. 

%As a methodology, I  combining linguistic intuition with statistical machine learning, especially probabilistic graphical models and deep learning methods. 


\section{Discourse Analysis} \label{discourse}

A well-written text is not merely a sequence of independent and isolated sentences, but instead a sequence of structured and related sentences. It addresses a particular topic, often covering multiple subtopics, and is organized in a coherent way that enables the reader to process the information. In discourse analysis we seek to uncover such underlying structures, which can support many downstream applications including machine translation, summarization and information extraction. In my PhD thesis, I proposed novel computational models for discovering the \emph{rhetorical} structure of texts, and the \emph{topical} and the \emph{dialogue} structures of written asynchronous conversations. My PhD thesis was financially supported by the \href{http://www.nserc-crsng.gc.ca/Students-Etudiants/PG-CS/BellandPostgrad-BelletSuperieures_eng.asp}{NSERC Alexander Graham Bell Canada Graduate Scholarship (CGS-D)}, which is awarded to the top-ranked PhD students across Canada. 


%In Spring 2014, I received the Alan J. Perlis SCS Student Teaching Award in recognition of my service; this award is granted to one student each year from the School of Computer Science at Carnegie Mellon "who has shown the highest degree of excellence and dedication" in teaching. This award recognized my service in designing and co-teaching a new interdisciplinary course ("Digital Literary and Cultural Studies") and in serving as TA for the core CS course of "Natural Language Processing."

%My approaches rely on computational methodologies from graph theory and probabilistic graphical models.

\subsection{Rhetorical Analysis} \label{subsec:codra}

Clauses in a sentence and sentences in a text are logically connected --- the meaning of one relates to that of the previous and the following ones. This logical relation between clauses is called the \emph{coherence structure} of the text. Different formal theories have been proposed to describe this structure. Rhetorical Structure Theory (RST) is perhaps the most influential one, which posits a tree-like discourse structure. For example, consider the discourse tree in Figure \ref{fig:DTexample} for the following text:


\begin {itemize}
\item[] \emph{But he added: ``Some people use the purchasers' index as a leading indicator, some use it as a coincident indicator. But the thing it's supposed to measure --- manufacturing strength --- it missed altogether last month."}   
\end{itemize}

\begin{figure}[htbp]
	\centering
		\includegraphics [height=40mm,width=138mm] {DT_less2.eps}
		\caption{A sample discourse tree. Horizontal lines indicate text segments; satellites are connected to their nuclei by curved arrows and two nuclei are connected with straight lines.}
		\label{fig:DTexample}
\end{figure}


The leaves of a discourse tree correspond to contiguous atomic text spans, called elementary discourse units or EDUs (six in the example). Adjacent EDUs are connected by coherence relations (e.g., \emph{Elaboration}, \emph{Contrast}), forming larger discourse units, which in turn are also subject to this relation linking. Discourse units linked by a relation are further distinguished based on their relative importance in the text: nuclei are the core parts of the relation while satellites are peripheral ones. For example, in Figure~\ref{fig:DTexample}, the satellite EDU ``--- manufacturing strength ---''  \emph{elaborates} the nucleus EDU ``But the thing it's supposed to measure'',   and two nuclei EDUs ``Some people use the purchasers index as a leading indicator'' and ``some use it as a coincident indicator'' \emph{contrast} each other. Conventionally, rhetorical analysis involves two subtasks: \Ni \textbf{discourse segmentation} is the task of breaking the text into a sequence of EDUs, and \Nii \textbf{discourse parsing} is the task of linking the discourse units (EDUs and larger units) into a labeled tree. 

Previous approaches to discourse parsing suffer from three key limitations: first, they typically model the structure and the labels of a discourse tree separately in a pipeline fashion, and also do not consider the sequential dependencies between the tree constituents; second, they typically apply greedy and sub-optimal parsing algorithms to build a tree; third, they do not discriminate between intra-sentential parsing (i.e., building the discourse trees for the individual sentences) and multi-sentential parsing (i.e., building the discourse tree for the whole document). 

In my PhD dissertation, I developed \href{http://alt.qcri.org/tools/discourse-parser/} {CODRA} -- a COmplete Discriminative framework for Rhetorical Analysis \cite{Shafiq_codra15,Shafiq13,Shafiq12}, which comprises a discourse segmenter and a discourse parser. CODRA addresses the above-mentioned limitations of existing parsers and to the best of my knowledge is still the state-of-the-art and most widely used tool for rhetorical analysis. The crucial component is the use of a probabilistic discriminative parsing model, expressed as a Dynamic Conditional Random Field (DCRF), to infer the probability of all possible tree constituents. By representing the structure and the relation of each tree constituent jointly and by explicitly capturing the sequential dependencies between tree constituents, the DCRF model does not make any independence assumption among these properties. CODRA uses the inferred (posterior) probabilities from the parsing models in a  probabilistic CKY-like bottom--up parsing algorithm, which is non-greedy and optimal. Furthermore, a simple modification of this parsing algorithm allows us to generate $k$-best parse hypotheses, that are later used in a \emph{tree kernel-based reranker}  to improve over the initial ranking using additional (global) features of the discourse tree as evidence \cite{Shafiq_EMNLP2014}. I made the \href{http://alt.qcri.org/tools/discourse-parser/} {source code} and a web-based \href{http://alt.qcri.org/demos/Discourse_Parser_Demo/} {demo} of the discourse parser publicly available for research purposes.\footnote{Available at http://alt.qcri.org/tools/discourse-parser/} 
 

%It  %The segmentation model not only achieves state-of-the-art performance, but also reduces the time and space complexities by using fewer features. 



%CODRA devises two different parsing components: one for intra-sentential parsing, the other for multi-sentential parsing. This provides for scalable, modular and flexible solutions that can exploit the strong correlation observed between the text structure and the discourse structure. A series of empirical evaluations over two very different data sets demonstrates that CODRA significantly outperforms existing approaches, often by a wide margin. 

\textbf{Future work:} I plan to investigate to what extent discourse segmentation and parsing can be performed jointly. I would also like to explore how CODRA performs on other genres like conversational (e.g., blogs) and evaluative (e.g., reviews) texts. To address the problem of limited annotated data in various genres, I am planning to develop an interactive version of CODRA that will allow users to fix the output of the system with minimal effort and let the system learn from that feedback.



\subsection{Topic Segmentation and Labeling in Asynchronous Conversations}

A discourse, whether it is a monologue or a conversation, exhibits a topic structure. For example, a news article about an earthquake may talk about the intensity, the damage, the aftershocks, and the casualties. Likewise, an email conversation about arranging a conference may discuss conference schedule, organizing committee, accommodation,  and registration. \textbf{Topic segmentation} refers to the task of grouping the sentences into a set of coherent topical segments, and \textbf{topic labeling} is the
task of assigning short descriptions to the topical segments to facilitate interpretations of the topics. 

%Beside the coherence structure, a text also has a topic structure --- it addresses a common topic, often covering multiple subtopics.   In other words, 

While extensive research has been conducted in topic segmentation for monologue and for synchronous dialogue (e.g., meetings), no-one had studied this problem for asynchronous conversations before me. Therefore, there was no reliable annotation scheme, no standard corpus, and no agreed-upon evaluation metrics available. Because of the asynchronous nature, topics in these conversations are often interleaved and do not change in a sequential way as they do in monologue and in synchronous dialogue. As a result, we do not expect models which have proved successful in these domains to be as effective, when directly applied to asynchronous conversations.  

%Therefore, the sequentiality constraint of topic segmentation in monolog and synchronous dialog does not hold in asynchronous conversation. As a result, we do not expect models which have proved successful in monolog or synchronous dialog to be as effective when directly applied to asynchronous conversation.
%Therefore, the sequentiality constraint of topic segmentation in monologue and in synchronous dialogue does not hold in asynchronous conversation. 

In my PhD dissertation, I presented two new \href{https://www.cs.ubc.ca/cs-research/lci/research-groups/natural-language-processing/bc3.html} {corpora} of email and blog conversations annotated with topics, and evaluated annotator reliability for the tasks using a new set of metrics, which were also used to evaluate the computational models. I also developed a complete \href{https://www.cs.ubc.ca/cs-research/lci/research-groups/natural-language-processing/Software.html} {computational framework} for performing topic segmentation and labeling in asynchronous conversations \cite{Shafiq13_Jair,Shafiq10}. For topic segmentation, I proposed two novel unsupervised models that exploit the fine-grained conversational structure beyond the lexical information. I also proposed a novel graph-theoretic supervised topic segmentation model that combines lexical, conversational and topic features. For topic labeling, I proposed two novel guided random walk models that respectively captures conversation specific clues from two different sources: the leading sentences and the fine-grain conversational structure. Empirical evaluation shows that the segmentation and the labeling performed by the best models outperform the state-of-the-art, and are highly correlated with human annotations. The corpora and the software were made publicly available for research purposes.\footnote{https://www.cs.ubc.ca/cs-research/lci/research-groups/natural-language-processing/Software.html} 

%To the best of my knowledge, this is the first comprehensive study to address these tasks in asynchronous conversation.

\textbf{Future work:} One interesting future direction would be to perform a more extrinsic evaluation of the framework. Instead of testing it with respect to human standards, it would be interesting to see how effective they are in supporting downstream applications, such as conversation summarization and other analytic tasks involving conversations. An example of the later is the \href{https://www.cs.ubc.ca/cs-research/lci/research-groups/natural-language-processing/ConVis.html}{ConVis} visual analytic system\footnote{https://www.cs.ubc.ca/cs-research/lci/research-groups/natural-language-processing/ConVis.html} for exploring blog conversations, which uses my framework in the backend.  


%I am also interested in transferring our approach to other similar domains by domain adaptation methods. I plan to work on both synchronous and asynchronous domains.

%\vspace{-0.1cm}
\subsection{Dialogue Act Recognition in Asynchronous Conversation}

Apart from the topic and the coherence structures, a conversational discourse (synchronous or asynchronous) also exhibits a dialogue structure; participants interact with each other by performing certain communicative acts like asking questions or requesting something, which are called \textbf{dialogue acts}. The two-part structures connecting two acts (e.g., \emph{Question-Answer, Request-Accept}) are called \textbf{adjacency pairs}. 
Previous work on dialogue act modeling has mostly focused on synchronous conversations, and the dominant approaches use supervised sequence taggers like linear-chain CRFs to capture the conversational dependencies (e.g., adjacency pairs) between the act types.    

%However, unlike synchronous conversations, consecutive turns in asynchronous conversations can be far apart in time and multiple turns can largely overlap. Furthermore, when messages are long, capturing sequential dependencies between the act types for sentence-level tagging becomes more challenging because the two components of adjacency pairs could be far apart in the sequence. Because of the asynchronous nature, the temporal order often lacks these dependencies. 


However, modeling conversational dependencies in asynchronous conversation is challenging, because the conversational flow often lacks sequential dependencies in its temporal order. For example, if we arrange the sentences as they arrive in the conversation, it becomes hard to capture any dependency between the act types because the two components of the adjacency pairs can be far apart in the sequence. This leaves us with one open research question: how to model the dependencies between sentences in a single comment and between sentences across different comments? %In this paper, we attempt to address this question by designing and experimenting with conditional structured models over arbitrary graph structure of the conversation.



In my PhD thesis, I proposed \emph{unsupervised} conversational models  \cite{Shafiq11b}.  First, I showed that like synchronous conversations, it is important for a conversational model in asynchronous conversations to capture the sequential dependencies between the act types. Then, I demonstrated that the conversational models, which are variants of unsupervised Hidden Markov Models (HMMs), learn better sequential dependencies when they are trained on the sequences extracted from the finer conversational structure compared to when they are trained on the temporal order of the sentences. Further investigation shows that the simple unsupervised HMM tends to find topic clusters in addition to dialogue act clusters. To address this problem, I proposed HMM+Mix model which not only explains away the topics, but also improves the act emission distribution by defining it as a mixture model. A part of this work was conducted at Microsoft Research Asia, for which I was given the \emph{``Microsoft Research Excellent Intern''} award.   

In a recent work \cite{Shafiq16ACL}, I propose a class of \emph{supervised} structured models in the form of Conditional Random Fields (CRF) defined over arbitrary graph structures of the asynchronous conversation. To surmount the problems with the bag-of-words type representations, my models use sentence representations encoded by a long short term memory (LSTM) recurrent neural model. The LSTM considers the word order and the compositionality of phrases while encoding a sentence vector. Empirical evaluation over three different datasets shows the effectiveness of this approach over existing ones: LSTMs provide better task-specific representations and the global joint model improves over local models. I have also released the \href{http://alt.qcri.org/tools/speech-act/}{source code and the datasets} for research purposes.\footnote{http://alt.qcri.org/tools/speech-act/}

\textbf{Future work:} I would like to couple CRF with LSTM, so that the LSTM can learn its parameters using
the global thread-level feedback. This would require the backpropagation algorithm to take error signals from the global inference algorithm (e.g., loopy belief propagation). We would also like to develop models for  conversations, where the conversational structure is given or extractable using the meta data, e.g., reply-to links and usage of quotations in email threads.


\subsection{Coherence Modeling} \label{coh-model}

Text analysis models that can distinguish a coherent from incoherent texts are known as coherence models. Such models have a range of applications in text generation, summarization, and coherence scoring. Inspired by the \emph{Centering Theory} \cite{Grosz_95}, Barzilay and Lapata \cite{} proposed the popular \textbf{entity grid} model of coherence. The model represents a text by a grid (Figure \ref{fig:egrid}) that captures how grammatical roles of different entities change from sentence to sentence. The grid is then converted into a feature vector containing probabilities of local entity transitions, which enables machine learning models to learn the degree of text coherence. A number of extensions (e.g., incorporating entity-specific features) of this basic entity grid model have been proposed by other researchers.  
%Extensions of this basic grid model incorporate entity-specific features (Elsner and Charniak

%% figure here...



While the entity grid and its extensions have been successful in many applications, they are limited in several ways. First, they use discrete representation for grammatical roles and other features, which prevents the model from considering sufficiently long transitions. Second, feature vector computation in existing models is decoupled from the target task, which limits the model’s capacity to learn task-specific features. In our recent work \cite{}, we propose a neural architecture for coherence assessment that can capture long range entity transitions along with arbitrary entity- specific features. Our model obtains generalization through distributed representations of entity transitions and entity features. We also present an end-to-end training method to learn task-specific high level features automatically in our model. Our evaluation on three different tasks show that  our model achieves state of the art results in all these tasks. We have released our \href{https://github.com/datienguyen/cnn_ coherence/}{source code} for research purposes.


\textbf{Ongoing work:} The above model was proposed for monologic texts (e.g., news articles) and only considers information regarding entities. In our ongoing work, we are extending this model in two different ways. First, we incorporate conversational structure into our model to extend it to asynchronous conversations (e.g., forums). %Neural IR paper.   


%we are incorporating other sources of coherence signals (e.g., rhetorical relations from the CODRA parser in Section \ref{subsec:codra}) in our model.     

%we would like to include other sources of information in our model. Our initial plan is to include rhetorical relations, which has been shown to benefit existing grid models (Feng et al., 2014). We would also like to extend our model to other forms of discourse, especially, asynchronous con- versations, where participants communicate with each other at different times (e.g., forum, email).
 
 

\section{Applications of Discourse Analysis} \label{discourse_app}

As mentioned before, discourse analysis has many applications. I am particularly interested in two of them: \Ni machine translation and \Nii text summarization. 


%After my PhD, I started working on the applications of discourse analysis. In particular, I am interested in two downstream applications: \Ni machine translation and \Nii text summarization. 


\subsection{Discourse for Machine Translation}

Among other applications of discourse, Machine Translation (MT) and its evaluation have received a resurgence of interest recently. Researchers now believe that MT systems should consider discourse phenomena that go beyond the current sentence to ensure consistency in the choice of lexical items or referring expressions, and the fact that source-language coherence relations are also realized in the target language. Automatic MT evaluation is an integral part of the process of developing and tuning MT systems. Reference-based evaluation metrics compare the output of a system to one or more human (reference) translations, and produce a similarity score indicating the quality of the translation. The initial MT metrics approached similarity as a shallow word $n$-gram matching between the translation and the reference, with a limited use of linguistic information. BLEU is the best-known metric in this family, which has been used for years. However, it has been shown that BLEU and akin metrics are insufficient and unreliable for high-quality translation output. 

Modeling discourse brings together the usage of higher-level linguistic information and the exploration of relations beyond the sentence level, which makes it a very attractive goal for MT and its evaluation. In particular, we believe that the semantic and pragmatic information captured in the form of RST trees (e.g., Figure \ref{fig:DTexample}) \Ni\ can yield better MT evaluation metrics, and \Nii\ can help develop discourse-aware MT systems that produce more coherent translations. In \cite{Shafiq_discoMT16,guzman-EtAl:ACL2014}, we have explored the first research hypothesis, i.e., \Ni. Specifically, we show that discourse information can be used to produce evaluation measures  that improve over the state-of-the-art in terms of correlation with human assessments. We conduct our research in four steps. 

First, we design a simple discourse-aware metric \qcril, which use sub-tree kernel to compare RST trees generated with CODRA. We show that this metric helps to improve a large number of MT evaluation measures at the segment-level and at the system-level. Second, we show that tuning the weights in the linear combination of metrics using human assessed examples is a robust way to improve the effectiveness of the \qcril\ metric significantly. Third, we conduct an ablation study which helps us understand which elements of the RST tree have the highest impact on the quality of the evaluation measure. Interestingly enough, the \emph{nuclearity} feature (i.e., the distinction between main and subordinate units) turns out to be more important than the discourse relation labels. Finally, based on these findings, we extend the tree-based representations and present the \discoparty\ metric, which make use of a combination of discourse tree representations and many other metrics. The resulting combined metric with tuned weights scored best as compared to human rankings at the WMT14 Metrics task, both at the system and at the segment levels. %The \href{https://github.com/Qatar-Computing-Research-Institute} {source code} and other resources of this work has been made publicly available for research purposes.\footnote{https://github.com/Qatar-Computing-Research-Institute}

\textbf{Future work:} I would like to explore the potential of discourse information for improving MT systems. My initial plan is to use discourse information to re-rank a set of candidate translations, where the challenge is to establish the links between the discourse structure of the source and that of the translated sentences, trying to promote translations that preserve discourse structure.


\subsection{Discourse for Text Summarization}

Another important application of discourse structures is text summarization. In my MSc dissertation, I investigated query-focused summarization approaches for answering complex questions (more on this later in Section \ref{summ}). A significant challenge faced by researchers in this field is how to produce summaries that are not only informative but also coherent. I believe discourse structures (e.g., topic, dialogue acts, rhetoric) can play an important role in this aspect.  In my ongoing work, I am investigating the utility of discourse structures for conversation (e.g., fora, emails) summarization. The main idea is to impose constraints based on discourse structures  in the content selection model to generate summaries that are informative as well as coherent. Discourse trees generated by CODRA allow us to perform content selection at the EDU level. This can yield better compression as compared to existing approaches, which operate at the sentence level.   



\section{Machine Learning for NLP and Data Mining} \label{application_ML}

Aside from discourse and its applications, I have also worked on developing effective machine learning  models for end-user applications. My interests lie in two important sub-fields of machine learning, deep learning and probabilistic graphical models, and their combination.   

%am also  interested in 

   


\subsection{Deep Learning}

In recent years there has been a growing interest in deep neural networks (DNNs) with application to myriad of NLP and data mining tasks. I have explored DNNs for a number of applications including machine translation and its evaluation, opinion analysis, disaster response, and health informatics.   


\subsubsection{Deep Learning for Machine Translation and its Evaluation}

\paragraph{Machine Translation:} A notably successful attempt on using neural networks for Machine Translation (MT) was made by \cite{Devlin_2014_acl}. They proposed a Neural Network Joint Model (NNJM), which augments streams of source with target $n$-grams and learns a neural model over vector representation of such streams. They achieve impressive gains with NNJM used as an additional feature in the decoder. In \cite{Shafiq_da_CL16,joty-EtAl:2015:EMNLP2}, we advance the state-of-the-art by extending NNJM for domain adaptation in order to leverage the huge amount of out-of-domain data coming from  heterogeneous sources. 

We carry out our research in two ways: \Ni we apply state-of-the-art domain adaptation techniques, such as mixture modeling and data selection using the NNJM, and \Nii we propose two novel methods to perform adaptation through instance weighting and weight readjustment in the NNJM framework. Our first method uses data dependent regularization in the loss function to perform (soft) data selection, while the second method fuses the in- and the out-domain models to readjust their parameters. Our evaluation on the standard translation tasks demonstrates that the adapted models outperform the non-adapted baselines and the deep fusion model outperforms the other neural adaptation methods as well as phrase-table adaptation techniques. We also demonstrate that our methods are complementary to the existing methods and together the models can achieve better translation quality. We released our \href{http://www.statmt.org/moses/} {source code} in moses open source platform.\footnote{http://www.statmt.org/moses/} %In our ongoing work, we are investigating the end-to-end deep learning framework for machine translation, called \href{http://104.131.78.120/} {neural MT}. 


%More specifically, we propose two novel extensions of NNJM for domain adaptation. The first model minimizes the cross entropy by regularizing the loss function with respect to the in-domain model. The regularizer gives higher weight to the training instances that are similar to the in-domain instances. The second model takes a more conservative approach by additionally penalizing instances that are similar to the out-of-domain. 


\paragraph{Machine Translation Evaluation:} 

In another front \cite{guzman2015-ACL,guzman_mteval_CSL16}, we presented a framework for machine translation evaluation using neural networks in a pairwise setting, where the goal is to select the better translation from a pair of hypotheses, given the reference translation. In this framework, lexical, syntactic and semantic information from the reference and the two hypotheses are embedded into small distributed vector representations, and fed into a multi-layer perceptron that models non-linear interactions between each of the hypotheses and the reference, as well as between the two hypotheses. We experiment with the benchmark datasets from the WMT Metrics shared task, on which we obtain the best results published so far, with the basic network configuration. We also perform a series of experiments to analyze and understand the contribution of the different components of the network. We evaluate variants and extensions including, among others: fine-tuning of the semantic embeddings, and sentence-based representations modeled with recurrent neural networks. The proposed framework is flexible and generalizable, allows for efficient learning and scoring, and provides an MT evaluation metric that correlates with humans on par with the state of the art.




\subsubsection{Deep Learning for Opinion Analysis}

Fine-grained opinion mining involves: \Ni identifying the opinion holder, \Nii identifying the target or aspect of the opinion, \Niii detecting opinion expressions, and \Niv measuring the intensity and sentiment of the opinion expressions. For example, in the sentence ``John says, the hard disk is very noisy'', John, the opinion holder, expresses a very negative opinion towards the target ``hard disk'' using the opinionated expression ``very noisy''. In \cite{liu-joty-meng:2015:EMNLP}, we propose a general class of models based on Recurrent Neural Network (RNN) and word embeddings, that can be successfully applied to fine-grained opinion mining tasks without any task-specific feature engineering effort.  

%We experiment with several important RNN architectures including Elman-RNN, Jordan-RNN, long short term memory (LSTM) and their variations. We acquire pre-trained word embeddings from several external sources to give better initialization to our RNN models. The RNN models then fine-tune the word vectors during training to learn task-specific embeddings. We also present an architecture to incorporate other linguistic features into RNNs.

Our results on the task of opinion target extraction show that word embeddings improve the performance of state-of-the-art CRF models, when included as additional features. They also improve RNNs when used as pre-trained word vectors and fine-tuning them on the task gives the best results. A comparison between models demonstrates that RNNs outperform CRFs, even when they use word embeddings as the only features. Incorporating simple linguistic features into RNNs improves the performance even further. Our best results with LSTM RNN outperform the top performing system on the Laptop dataset and achieve the second best on the Restaurant dataset in SemEval-2014. We made our \href{https://github.com/ppfliu/opinion-target} {source code} publicly available for research.\footnote{https://github.com/ppfliu/opinion-target}


\subsubsection{Deep Learning for Sentence Representation}

Vector representation of sentences is important for many text processing tasks that involve clustering, classifying, or ranking sentences. Recently, distributed representation of sentences learned by neural models from unlabeled data has been shown to outperform the traditional bag-of-words representation. However, most of these learning methods  consider only the content of a sentence and disregard the relations among sentences by and large.  In a recent work \cite{Tanay_WWW_17}, we propose a series of novel models for learning latent representations of sentences (i.e., Sen2Vec) that consider the content of a sentence as well as inter-sentence relations. We first represent the inter-sentence relations with a language network and then use the network to induce contextual information into the content-based Sen2Vec models. Two different approaches are introduced to exploit the information in the network.  Our first approach \emph{retrofits} (already trained) Sen2Vec vectors with respect to the network in two different ways: \Ni using the adjacency relations of a node, and \Nii using a stochastic sampling method which is more flexible in sampling neighbors of a node. The second approach uses a regularizer to encode the information in the network into the existing Sen2Vec model. Experimental results show that our proposed models outperform existing methods in three fundamental tasks --- \emph{classification}, \emph{clustering}, and \emph{ranking}, demonstrating the effectiveness of our approach. 


%The models leverage the computational power of multi-core CPUs to achieve fine-grained computational efficiency. We make our code publicly available upon acceptance.


%Leveraging this important information source requires new methods and technologies to be developed to facilitate data-driven research for sleep and activity patient-recommendations.

%We compare the deep learning models with those build using classical approaches, i.e. logistic regression, support vector machines, random forest and adaboost. Secondly, we employ the advantage of deep learning with its ability to handle high dimensional datasets. We explore several deep learning models on the raw wearable sensor output without performing HAR or any other feature extraction.

%Our results show that using a convolutional neural network on the raw wearables output  improves the predictive value of sleep quality from physical activity, by an additional 8\% compared to state-of-the-art non-deep learning approaches \cite{arXivRAHAR}, which itself shows a 15\% improvement over current practice \cite{TeresaConvo}. 

\subsubsection{Deep Learning for Health Informatics}

Sleep quality is critical for maintaining physical, emotional and mental wellbeing. Although sleep and physical activity are known to be correlated, their relationship is not yet fully understood. The increasing popularity of actigraphy and wearable devices provides a unique opportunity to understand this relationship. In our recent work \cite{Aarti16_JMU,Aarti17_SDM}, we explore deep learning models for sleep quality prediction using actigraphy data. In one setting, we first perform human activity recognition (HAR) on raw sensor data, and then feed HAR output into both conventional and deep learning models to perform sleep quality prediction. In the other setting, we employ several deep learning models directly on the raw wearable sensor data without performing HAR or any other feature extraction.

Our results show that using a time-batched LSTM RNN on the raw wearables data improves the sleep quality prediction by an additional $10\%$ with an overall AUC of $0.97$ compared to the state-of-the-art non-deep learning approaches, which itself shows a $15\%$ improvement over the current clinical practice. Moreover, utilizing deep learning on raw data eliminates the need for data pre-processing and simplifies the overall workflow to analyze actigraphy data for sleep and physical activity research. From an application impact perspective, the proposed approach promises a very high-fidelity screening test for sleep disorders directly from wearables data, potentially replacing the need for an inconvenient and expensive visit to a sleep laboratory for an evaluation.




%Our results show that a \emph{convolutional neural network} on the raw wearables output  improves the predictive value of sleep quality by an additional 8\% compared to state-of-the-art non-deep learning approaches, which itself shows a 15\% improvement over current practice. Moreover, utilizing deep learning on raw data eliminates the need for data pre-processing and simplifies the overall workflow to analyze actigraphy data for sleep and physical activity research. 


 

%In this paper we explore the use of deep learning to build sleep quality prediction models based on actigraphy data. We first use deep learning as a pure model building device by performing human activity recognition (HAR) on raw sensor data, and using deep learning to build sleep prediction models. We compare the deep learning models with those build using classical approaches, i.e. logistic regression, support vector machines, random forest and adaboost. Secondly, we employ the advantage of deep learning with its ability to handle high dimensional datasets. We explore several deep learning models on the raw wearable sensor output without performing HAR or any other feature extraction.

%We compare the performance of several deep learning models such as recurrent neural networks and convolutional neural networks with those built using classical algorithms such as support vector machines and random forests in two different settings. 



\subsubsection{Deep Learning for Crisis Computing}

During the onset of a crisis situation (e.g., earthquake), rapid analysis of messages posted on microblogging platforms such as Twitter can help humanitarian organizations gain situational awareness, learn about urgent needs, and to direct their decision-making processes accordingly. However, time-critical analysis of such big crisis data brings challenges to machine learning techniques, especially to supervised learning methods. The scarcity of labeled data, particularly in the early hours of a crisis, delays the learning process. Traditional approaches use batch learning with hand engineered features like cue words and TF-IDF vectors. This approach has three major limitations. First, in the beginning of a disaster situation, there is no labeled data available for training for that particular event. Later, the labeled data arrives in minibatches depending on the availability of volunteers. Due to the discrete word representations and the variety across events, traditional classification models perform poorly when trained on previous (out-of-event) events. Second, training a classifier from scratch every time a new minibatch arrives is infeasible. Third, extracting right features for each disaster related classification task is time consuming and requires domain knowledge.   

Deep neural networks (DNNs) are ideally suited for disaster response with big crisis data. They are usually trained with online learning and have the flexibility to adaptively learn from new batches of labeled data without requiring to retrain from scratch. Due to their distributed word representation, they generalize well and make better use of the previously labeled data from other events to speed up the classification process in the beginning of a disaster. DNNs
obviate the need of manually crafting features and automatically learn latent features as distributed dense vectors, which generalize well.

In \cite{Dat16_bigData},  we proposed convolutional neural networks (CNN) for the classification tasks in a disaster situation. CNN captures the most salient $n$-gram information by means of its convolution and max-pooling operations. On top of the typical CNN, we propose an extension that combines multilayer perceptron with a CNN. We present a series of experiments using different variations of the training data -- event data only, out-of-event data only and a concatenation of both. Experiments are conducted for binary (\emph{useful} vs. \emph{not useful}) and multi-class (e.g., \emph{donations}, \emph{sympathy}, \emph{casualties}) classification tasks. Empirical evaluation shows that our CNN models outperform non-neural models by a wide margin in both classification tasks in all scenarios. In the scenario of no event data, the CNN model shows substantial improvement of up to 10 absolute points over several non-neural models. Our variation of the CNN model with multilayer perceptron performed better than its CNN-only counter part. Another finding is that blindly adding out-of-event data either drops the performance or does not give any noticeable improvement over the event only model. To reduce the negative effect of large out-of-event data and to make the most out of it, we apply two simple domain adaptation techniques -- \Ni weight the out-of-event labeled tweets based on their closeness to the event data, \Nii select a subset of the out-of-event labeled tweets that are correctly labeled by the event-based classifier. Our results show that the later results in better classification model. We have also released our \href{https://github.com/CrisisNLP/deep-learning-for-big-crisis-data}{source code} for the crisis computing research community.\footnote{https://github.com/CrisisNLP/deep-learning-for-big-crisis-data} 



%The current state-of-the-art classification methods require a significant amount of labeled data specific to a particular event for training plus a lot of feature engineering to achieve best results. In this work, we introduce neural network based classification methods for binary and multi-class tweet classification task. We show that neural network based models do not require any feature engineering and perform better than state-of-the-art methods. In the early hours of a disaster when no labeled data is available, our proposed method makes the best use of the out-of-event data and achieves good results.  


 % However, the volume of tweets is huge, about $350$ thousand tweets per minute. Filtering them using keywords helps cut down this volume to some extent. But, identifying {\it useful} tweets that responders can act upon cannot be achieved using only keywords because a large number of tweets may contain the keywords but are of limited utility for the responders. The best-known solution to address this problem is to use supervised classifiers that would separate useful tweets from the rest and classify them accordingly into useful categories (e.g., \emph{Donations, Infrastructure damage}).  

%We present a series of experiments using different variations of the training data -- in-event, out-of-event and in- plus out-of-event. We empirically demonstrate that the neural models consistently outperform the non-neural offline and online models by a wide margin (over $10\%$ absolute). We further provide extensive evaluation of the neural models by comparing random initialization with pre-trained word vectors, and the effect of fine tuning on pre-trained word vectors. This is a joint work with the United Nations (UN) and we make most of our \href{https://github.com/qcri-social/AIDR} {resources} freely  available.\footnote{https://github.com/qcri-social/AIDR}  %We make our code available to the research community.  


%\textbf{To do: have to mention the NSF grant} 

%We have recently filed a proposal for NSF grant to study this problem.     


%However, time-critical analysis of live social media streams brings challenges to machine learning techniques, especially the ones that use supervised learning. Scarcity of labeled data, particularly in the early hours, delays the machine learning process. 

%In this work, we show the usefulness of past labeled data to train machine learning classifiers to identify informative and uninformative tweets from Twitter. Specifically, in comparison to the traditional offline learning mechanisms used, we present deep neural network models that \Ni are based on an online learning algorithm, and, \Nii do not require feature engineering. The online nature of the neural network models best suit the disaster response situation where labeled data comes in small batches. The classification results show that neural models are significantly better than the non-neural models and achieve about 10\% absolute improvement over them.

\subsection{Probabilistic Graphical Models}

Probabilistic Graphical Models (PGMs) allow us to define arbitrary joint distributions compactly by exploiting the interactions among the variables, which is necessary for modeling complex dependencies in various NLP and data mining tasks. Apart from the PGMs proposed for discourse analysis tasks in Section \ref{discourse}, I proposed novel PGMs for Community Question Answering (cQA) \cite{joty-EtAl:2015:EMNLP1,Shafiq16_cqa_naacl,Shafiq16_cqa_tacl}.

In cQA, three tasks are of special relevance when a user poses a new question to the website: \Na determine whether a comment within a question-comment thread is a good answer to the question of that thread  (i.e., \emph{answer goodness}), \Nb find related questions to the new question (i.e., \emph{question-question similarity}), and \Nc find relevant answers to the new question (i.e., \emph{answer selection}). These tasks are interrelated as the information needed to answer a new question is usually found in the good comments of highly related questions.

%, and \Niii find potential users who may provide good answers to the new question (i.e., \emph{user selection}) \cite{horowitz2010anatomy,hsieh2009mimir}. Another task that can help solving other tasks better is determining whether a comment within a question-comment thread is a good answer to the question of that thread (i.e., \emph{answer goodness}). 

In \cite{joty-EtAl:2015:EMNLP1,Shafiq16_cqa_naacl}, I focused on task \Na, i.e., classifying comments of an answer-thread as \emph{good} vs. \emph{bad} answers with respect to the thread question. The traditional approach learns a local classifier and uses it to predict for each comment separately. However, this approach ignores the structure in the answer-thread. I approached the task with a global inference process to exploit the information of all comments in the answer-thread in the form of a fully-connected graph. I proposed two novel joint learning models that are on-line and integrate inference within learning. The first one jointly learns two \emph{node}- and \emph{edge}-level MaxEnt classifiers with stochastic gradient descent and integrates the inference step with loopy belief propagation. The second model is an instance of fully connected pairwise CRFs (FCCRF), which performs a global normalization of the functions. The FCCRF model significantly outperforms all other approaches and yields the best results on the task to date. Crucial elements for its success are the global normalization and an Ising-like edge potential.    

In a more recent work \cite{Shafiq16_cqa_tacl}, I consider solving tasks \Nb and \Nc jointly with the help of task \Na in a joint multi-task learning framework. My approach has two steps. First, a DNN in the form of a feed-forward neural network is trained to solve each of the three individual tasks, and the task-specific hidden layer activations are taken as embedded feature representations to be used in the second step. Then, a structured conditional  model, a conditional random field (CRF), uses these embeddings and performs joint learning with global inference to exploit the dependencies between the different tasks. 







% The experimental results show that DNNs alone achieve competitive results when trained to produce the embeddings. Furthermore, the structured model is able to effectively make use of the embeddings and the dependencies between the tasks to improve results significantly and consistently across a variety of evaluation metrics, thus showing the complementarity of DNNs and structured learning.



Our system has been deployed in a real cQA \href{http://www.qatarliving.com/betasearch} {forum site}. We have also designed and implemented a web-based interactive \href{http://www.qatarliving.com/betasearch}{cQA interface}, which has been evaluated with real forum users (see \cite{Hoque:Joty:17}).


%Online community fora (e.g., Stack Exchange) are quite open, allowing anybody to ask and anybody to answer a question, which makes them very valuable sources of information. Yet, this same democratic nature resulted in some questions accumulating a large number of answers, many of which are of low quality. While nowadays online fora are typically searched using standard search engines that index entire threads, this is not optimal, as it can be very time-consuming for a user to go through the entire thread and make sense out of it. Thus, the creation of automatic cQA systems, which could provide efficient and effective ways to find good answers has received a lot of attention recently.
%We 

\subsection{Combining Deep Learning and Probabilistic Graphical Models}
A key strength of deep learning approaches is their ability to learn nonlinear interactions between underlying features through specifically designed hidden layers, and also to learn the features (e.g., word vectors) automatically. In the case of unstructured output problems, this capability has led to gains in many tasks. Deep neural methods are also powerful for structured output problems. Existing work has mostly relied on recurrent or recursive architectures, which can propagate information through hidden layers, but as a result disregard the modeling strength of PGMs, which use global inference to model consistency in the output structure (i.e., class labels of all nodes in a graph). 


My current research goal is to combine these two types of models in order to exploit their respective strengths in modeling the input and the output. In my very recent work on speech act recognition (SAR) \cite{Shafiq16ACL} and on community question answering (cQA) \cite{Shafiq16_cqa_tacl}, I demonstrate the effectiveness of this marriage of DNNs and PGMs. In both problems, a DNN is first used to encode task-specific embeddings and to perform local classifications which are then ''reconciled'' in a conditional structured modeling framework, i.e., conditional random fields. However, this integration was done as a two-step process. To make the model more effective, I would like to couple PGMs with DNNs, so that the DNNs can be optimized directly on the global (structured) output. 

%To showcase the effectiveness of this approach, I plan to investigate a joint multi-task deep learning framework to solve the full cQA task, i.e., finding good answers to newly-asked questions by exploiting the similarities between the new question and the existing questions in the database and also the relations between the comments in and across different threads.





\subsection{Early Work on Unsupervised Models for Question Answering \& Summarization} \label{summ}

In my M.Sc. thesis, I built an open-domain question answering (QA) system, where I investigated unsupervised methods to automatically answer both simple and complex questions. Simple questions (e.g., ``Who is the president of USA?'') require small snippets of text as answers and are easier to answer than complex questions (e.g., ``Describe the after-effects of cyclone Cindy?'') which entail richer information needs and require synthesizing information from multiple documents.

My work on answering complex questions was published in a JAIR article \cite{Chali:2009:CQA} and in one conference paper \cite{Chali:2008}. I approached the task as a query-focused multi-document summarization and employed an extractive approach to select a subset of the original sentences as the answer. In particular, I experimented with one simple vector space model and two statistical unsupervised models for computing the importance of the sentences. The performance of these approaches depends on the features used and the weighting of these features. I extracted different kinds of informative features for each sentence, and use a gradient descent search to learn the feature-weights from a development set. I first showed that the tree kernel features based on the syntactic and shallow semantic trees of the sentences improve the performance of these models significantly, then I showed that with a large feature set and the optimal feature-weights, my unsupervised models perform as good as state-of-the-art systems with the advantage of not requiring any human annotated data for training. In a separate but related work \cite{Chali:acl:2008}, I show that the syntactic and shallow semantic tree kernels can also improve the performance of the random walk model for answering complex questions. 


My approach to answering simple questions was based on question classification and document tagging. Question classification extracts information about how to answer the question (i.e., answer types), and document tagging extracts useful information from the source documents, which are used in finding the answer. To classify the questions based on their answer types (e.g., person, location), I use rules developed manually by analyzing a large set of questions. I employ different off-the-shelf
taggers to identify useful information in the documents. In the TREQ-07 QA evaluation, my system
was ranked 4th and 6th among 51 participants in the factoid and list type questions, respectively.


\section{Future Research Directions} \label{future}

Future directions that are directly related to my previous work are described throughout Sections \ref{discourse}, \ref{discourse_app}, and \ref{application_ML}. In this section, I highlight some of the other ideas that I aspire to explore.




\subsection{Other Discourse Analysis Tasks in Asynchronous Conversation}

In my discourse-related research, so far I have explored discourse parsing, topic modeling, and dialogue act recognition (Section \ref{discourse}). While discourse parsing can be applied to any form of discourse, the other two address asynchronous conversations. There are other discourse analysis tasks in asynchronous conversations that did not get any attention yet. Two tasks that I am particularly interested in are: \Ni coreference resolution, and \Nii coherence modeling. Although many researchers have studied these tasks in monologue, %\cite{Barzilay05ACL,Durrett13}, 
none has yet addressed them in asynchronous domains. As mentioned before, asynchronous conversation poses a new set of challenges for computational models, and methods designed for monologues often do not work well when applied directly to asynchronous conversation. In my future work I would like to investigate these two tasks for asynchronous conversation by exploiting their conversational structure in joint structured models.      

\subsection{NLP Applications}

I would like to continue exploring new machine learning methods for NLP applications, in particular neural methods for machine translation and summarization. Thus far, my approach to summarization has been \emph{extractive}, where informative sentences are selected to form an abridged version of the source document(s). This approach has been by far the most popular in the field of summarization, largely because it does not require to generate novel sentences. Another approach which is getting more attentions recently is \emph{abstractive} summarization, where the goal is to generate novel texts. Thus, abstractive approach is expected to take us closer to human-style summarization. Very recently, the end-to-end deep learning framework for machine translation (i.e., \href{http://104.131.78.120/} {neural MT}) \cite{bahdanau:ICLR:2015} has been used for generating abstractive summaries \cite{rush2015neural}. While the framework works well for generating short summaries (e.g., headlines), generating longer summaries is still a big challenge. I plan to investigate neural MT for both machine translation and abstractive summarization. 



\subsection{Group Mining in Social Networks}

Beside NLP and its applications, I would be eager to work on data mining tasks in social networks. One particular problem that I am interested in is understanding how groups or teams in social networks behave --- group formation and evolution process --- which has many real world applications.  Groups are best represented with hypergraphs, where hyperedges are used to represent the groups and their interactions. Our proposal is to come up with a \emph{hyperedge2vec} model for learning vector representations for hyperedges, which can then be used in group-related prediction tasks. On the prediction side, our aim is to come up with a conditional structured model to exploit the interactions between sub-groups. We have recently filed a proposal for NSF grant to study this problem \cite{NSF_group}.

%\textbf{To do: have to mention the NSF grant}

     



\bibliography{shafiq_sor}
\bibliographystyle{plain}

\end{document}
